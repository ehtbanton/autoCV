
%----------------------------------------------------------------------------------------
%	DOCUMENT DEFINITION
%----------------------------------------------------------------------------------------

% article class because we want to fully customize the page and not use a cv template
\documentclass[a4paper,10pt]{article}
\usepackage[T1]{fontenc}
\renewcommand{\normalsize}{\fontsize{10}{12}\selectfont}

%----------------------------------------------------------------------------------------
%	FONT
%----------------------------------------------------------------------------------------

% % fontspec allows you to use TTF/OTF fonts directly
% \usepackage{fontspec}
% \defaultfontfeatures{Ligatures=TeX}

% % modified for ShareLaTeX use
% \setmainfont[
% SmallCapsFont = Fontin-SmallCaps.otf,
% BoldFont = Fontin-Bold.otf,
% ItalicFont = Fontin-Italic.otf
% ]
% {Fontin.otf}

%----------------------------------------------------------------------------------------
%	PACKAGES
%----------------------------------------------------------------------------------------
\usepackage{url}
\usepackage{parskip}

% Custom commands for email and phone links
\newcommand{\emaillink}[1]{\href{mailto:#1}{#1}}
\newcommand{\phonelink}[2]{\href{tel:#1}{#2}} 	

%other packages for formatting
\RequirePackage{color}
\RequirePackage{graphicx}
\usepackage[usenames,dvipsnames]{xcolor}
\usepackage[scale=0.85, top=2cm, bottom=2cm, left=2cm, right=2cm]{geometry}

%tabularx environment
\usepackage{tabularx}

%for lists within experience section
\usepackage{enumitem}

% centered version of 'X' col. type
\newcolumntype{C}{>{\centering\arraybackslash}X}

%to prevent spillover of tabular into next pages
\usepackage{supertabular}
\newlength{\fullcollw}
\setlength{\fullcollw}{0.47\textwidth}

%custom \section
\usepackage{titlesec}				
\usepackage{multicol}
\usepackage{multirow}

%CV Sections inspired by: 
%http://stefano.italians.nl/archives/26
\titleformat{\section}{\Large\scshape\raggedright}{}{0em}{}[\titlerule]
\titlespacing{\section}{0pt}{8pt}{4pt}

%for publications
\usepackage[style=authoryear,sorting=ynt, maxbibnames=2]{biblatex}

%Setup hyperref package, and colours for links
\usepackage[unicode, draft=false]{hyperref}
\definecolor{linkcolour}{rgb}{0,0.2,0.6}
\hypersetup{
    colorlinks=true,
    breaklinks=true,
    urlcolor=linkcolour,
    linkcolor=linkcolour,
    pdfborder={0 0 0},
    pdfcreator={pdflatex},
    pdfproducer={pdflatex}
}
\addbibresource{citations.bib}
\setlength\bibitemsep{1em}

%for social icons
\usepackage{fontawesome5}

%debug page outer frames
%\usepackage{showframe}


% job listing environments
\newenvironment{jobshort}[2]
    {
    \begin{tabularx}{\linewidth}{@{}l X r@{}}
    \textbf{#1} & \hfill &  #2 \\[2pt]
    \end{tabularx}
    }
    {
    }

\newenvironment{joblong}[2]
    {
    \begin{tabularx}{\linewidth}{@{}l X r@{}}
    \textbf{#1} & \hfill &  #2 \\[2pt]
    \end{tabularx}
    \begin{minipage}[t]{\linewidth}
    \begin{itemize}[nosep,after=\strut, leftmargin=1em, itemsep=1pt,label=--]
    }
    {
    \end{itemize}
    \end{minipage}    
    }



%----------------------------------------------------------------------------------------
%	BEGIN DOCUMENT
%----------------------------------------------------------------------------------------
\begin{document}

% non-numbered pages
\pagestyle{empty} 

%----------------------------------------------------------------------------------------
%	TITLE
%----------------------------------------------------------------------------------------

% \begin{tabularx}{\linewidth}{ @{}X X@{} }
% \huge{Your Name}\vspace{2pt} & \hfill \emoji{incoming-envelope} email@email.com \\
% \raisebox{-0.05\height}\faGithub\ username \ | \
% \raisebox{-0.00\height}\faLinkedin\ username \ | \ \raisebox{-0.05\height}\faGlobe \ mysite.com  & \hfill \emoji{calling} number
% \end{tabularx}

\begin{tabularx}{\linewidth}{@{} C @{}}
\Huge{Anton May} \\[8pt]
\href{https://github.com/ehtbanton}{\textbf{\raisebox{-0.05\height}\faGithub\ ehtbanton}} \ $|$ \
\href{https://www.linkedin.com/in/antonpmay}{\textbf{\raisebox{-0.05\height}\faLinkedin\ antonpmay}} \ $|$ \
% \href{https://mysite.com}{\raisebox{-0.05\height}\faGlobe \ mysite.com} \ $|$ \
\raisebox{-0.05\height}\faEnvelope \ anton.may@new.ox.ac.uk \ $|$ \
\raisebox{-0.05\height}\faMobile \ +447849262611 \\
\end{tabularx}

\vspace{6pt}

\begin{center}
\small{Final-year Oxford engineer who easily gets engrossed in an exciting new problem. I place care towards evaluating poorly defined failure modes so I can build reliable and safe non-deterministic end-to-end systems.}
\end{center}

\vspace{4pt}

%----------------------------------------------------------------------------------------
% EXPERIENCE SECTIONS
%----------------------------------------------------------------------------------------


%Experience
\section{Work Experience}

\begin{joblong}{Software Engineer - Big Data Institute, Oxford}{Jun 2025 - Oct 2025}
\item Revamped \href{http://141.147.64.20:3000/zarr}{\textbf{AIDA-3D}}, a 3D medical image viewer, from a 3-year-old untouched demo into a useful tool for arbitrary web-hosted data aligned with the OME-Zarr v0.5 spec
\item Reverse-engineered undocumented data pipelines with no external guidance
\item The project group are seeking £2M grant funding, with reference to how the work I'm doing in my 4th year project could be integrated into AIDA-3D to make it a text-searchable 3D image database.
\end{joblong}
\vspace{2pt}
\begin{joblong}{Software Development Intern - Valor Carbon}{Aug 2025 - Sep 2025}
\item Built an automated document generation tool using LLMs and RAG; deployed securely on Oracle Cloud
\item Learned that with agentic systems, it's best to focus on practical automation over forced implementation
\end{joblong}
\vspace{2pt}
\begin{joblong}{LLM Performance Evaluator - Outlier}{Dec 2024 - Apr 2025}
\item Evaluated LLM outputs for accuracy and clarity; provided feedback for C++ and mathematical reasoning
\end{joblong}
\vspace{2pt}
\begin{joblong}{Academical Clerk - New College, Oxford}{Oct 2022 - present}
\item Performing daily as a professional singer in a globally renowned choir alongside my studies.
\end{joblong}

%Projects
\section{Projects}

\begin{tabularx}{\linewidth}{ @{}l r@{} }
\textbf{(4YP) Analysing radiology reports using large language models} & \hfill Oct 2025 - present \\[2pt]
\multicolumn{2}{@{}X@{}}{My final year project. Building a tool to automatically label radiology reports using a fine-tuned local LLM, unlocking a huge untapped resource of training data for medical imaging models. Designing a checking system to filter errored labels and benchmarking against state-of-the-art approaches.}  \\
\end{tabularx}
\vspace{2pt}
\begin{tabularx}{\linewidth}{ @{}l r@{} }
\textbf{ThirdEye - Navigation for the Visually Impaired} & \hfill Nov 2025 - present \\[2pt]
\multicolumn{2}{@{}X@{}}{A discreet IoT wearable (baseball cap) which connects to a mobile companion app to create stereoscopic depth maps, perform YOLO object recognition, and call GPT-4-mini to build a rich understanding of the user's environment. We've just finished our second prototype and submitted to Microsoft Imagine Cup.}  \\
\end{tabularx}
\vspace{2pt}
\begin{tabularx}{\linewidth}{ @{}l r@{} }
\textbf{Exammer - Cofounder} & \hfill Sep 2025 - present \\[2pt]
\multicolumn{2}{@{}X@{}}{The verified knowledge portfolio for the post-AI world (\href{https://exammer.co.uk}{\textbf{exammer.co.uk}}). A platform where anyone can get credit for what they study, featuring a phone-call style AI interviewer with an interactive whiteboard which marks you as you go, and RAG lookup on conversations to prove a user's subject-specific work to employers.}  \\
\end{tabularx}
\vspace{2pt}
\begin{tabularx}{\linewidth}{ @{}l r@{} }
\textbf{Quantext - Quantum Dice Competition Winner} & \hfill Nov 2025 \\[2pt]
\multicolumn{2}{@{}X@{}}{A QUBO-based RAG algorithm addressing a known flaw in LLM context retrieval on massive, cluttered datasets. Built for Quantum Dice's ORBIT quantum compute platform. Won £1000; only undergrads in top 3.}  \\
\end{tabularx}
\vspace{2pt}
\begin{tabularx}{\linewidth}{ @{}l r@{} }
\textbf{Intelligent Musical Lighting System - Project Leader} & \hfill Oct 2024 - May 2025 \\[2pt]
\multicolumn{2}{@{}X@{}}{Led a 4-person team building an AI system for automated concert lighting based on musical feature analysis.}  \\
\end{tabularx}

\vspace{6pt}

%----------------------------------------------------------------------------------------
%	EDUCATION
%----------------------------------------------------------------------------------------
\section{Education}

\begin{joblong}{MEng Engineering Science, University of Oxford (predicted 2:1)}{Oct 2022 - Jul 2026}
\item 83\% in Engineering Computation (3rd highest in year) - programming project creating a tool to detect the tempo of a piece of music, and comparing its performance to an equivalent LLM-written tool.
\end{joblong}

\vspace{6pt}

%----------------------------------------------------------------------------------------
%	SKILLS
%----------------------------------------------------------------------------------------
\section{Skills}
\begin{tabularx}{\linewidth}{@{}l X@{}}
Languages &  \normalsize{Python, JavaScript/TypeScript, C++, Rust, MATLAB}\\[3pt]
ML/AI &  \normalsize{PyTorch, OpenCV, RAG systems, local LLMs, MCPs for agentic AI}\\[3pt]
Web \& Infrastructure &  \normalsize{Azure, Oracle Cloud, Google Cloud, CI/CD, distributed systems}\\[3pt]
Other &  \normalsize{Git, 3D graphics, medical imaging, IoT, quantum computing}\\
\end{tabularx}


\end{document}
